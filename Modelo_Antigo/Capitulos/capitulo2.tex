%%%%%%%%%%%%%%%%%%%%%%%%%%%%%%%%%%%%%%%%%%%%%%%%%%%%%%%%%%%%%%%%%%%%%%%%%%%%%%%
% CAP�TULO 2
\chapter{FUNDAMENTA��O TE�RICA}  


\section{Teorema de Bayes} \label{teoremaBayes}

\section{Classificadores Bayesianos}
Os classificadores Bayesianos surgiram com base no Teorma de Bayses, que foi apresentado na se��o \ref{teoremaBayes}. 
Segundo XXXXX, por ser um classificador probabil�stico simples, asssumindo que todas as caracter�sticas s�o indepentes entre si, o classificador Naive Bayes � bastante utilizado e conhecido na minera��o de dados.
O algoritmo do classificador Naive Bayes funciona da seguinte forma:
Dado um conjunto de treinamento \simbolo{$Tr$}{comprimento de onda}



\section{K-Dependence Bayesian Classifier (KDBC)} \label{kdbc}

\section{Problema}

Como visto na se��o \ref{kdbc} o poss�vel n�mero de modelos gerados pelo KDBC para a representa��o do classificador � determinado pela equa��o X. A equa��o cresce exponencialmente em fun��o do n�mero de atributos e do n�mero de k-dependencias. 
Assim, com o crescimento do n�mero de atributos ou da exig�ncia de um n�mero maior de k-dependencias, gerar todos os modelos ou escolher o melhor modelo para o problema torna-se computacionalmente muito custoso.


